\documentclass[11pt, oneside]{article}   	% use "amsart" instead of "article" for AMSLaTeX format
\usepackage{geometry}                		% See geometry.pdf to learn the layout options. There are lots.
\geometry{letterpaper}                   		% ... or a4paper or a5paper or ... 
%\geometry{landscape}                		% Activate for for rotated page geometry
%\usepackage[parfill]{parskip}    		% Activate to begin paragraphs with an empty line rather than an indent
\usepackage{graphicx}				% Use pdf, png, jpg, or eps� with pdflatex; use eps in DVI mode
								% TeX will automatically convert eps --> pdf in pdflatex		
\usepackage{amssymb}

\title{Statement of Purpose}
\author{Shankar Kulumani}
\date{}							% Activate to display a given date or no date

\begin{document}
\maketitle
%\section{}
%\subsection{}

On the first day of Astro 310, the introductory astronautics course at the US Air Force Academy, the professor asked bluntly, "What have you learned so far?"
Some stated that they knew trigonometry; to which the professor replied "Well, that brings you up to about the year 300 B.C."
Another would reply that we had all learned calculus as well, which would be answered with another blunt statement of "You're now all up to about the 1700s."
It was at this point where I finally realized that the wave of human knowledge continually moves forward and I want to lead it.

I earned my Bachelor�s of Science in Astronautical Engineering from the US Air Force Academy.  
My education at the Academy exposed me to this wide field and I became very interested in control systems theory, dynamics, and orbit mechanics.  
I found the ability to model, predict, and control the output of a system to be incredibly fascinating.  
I believe the knowledge of control theory can be applied to a wide variety of problems as well as issues well outside the realm of Astronautics. 

As a member of the United States Air Force I�ve been exposed to the program management aspects of space systems.  
 
I hope to further my understanding of control and dynamics and continue research in this field.    
My interests in these fields have led me to applying to Purdue University.  
As a full time member of the Air Force a quality distance learning program in this field was very important to me.  
Purdue University allows me to continue my education and pursue my dreams while still working full time for the Air Force.  
I believe that the Aeronautics and Astronautics Department at Purdue University is the ideal place to complete my graduate degree. 
The expert faculty and highly regarded program make Purdue University a excellent opportunity to continue my studies.  
I can directly apply the information learned in my graduate studies to the current issues on operational systems for the Air Force. 
I plan applying the knowledge from this graduate degree to further advance current space systems in the Air Force.  
After my career in the Air Force I hope to pursue a doctoral degree and become a professor or work for the private industry.  
The combination of a world class education and applicable experience in a space related field would prove useful to a professor.      




\end{document}  